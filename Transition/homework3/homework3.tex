\documentclass{article} 

\usepackage{fancyhdr}
\usepackage[english]{babel}

\usepackage{fullpage}
\usepackage[margin = .75 in]{geometry}
\usepackage[leqno]{amsmath}
\usepackage{amsfonts}
\usepackage{amssymb}
\usepackage{amsthm}
\usepackage{amssymb}
\usepackage[all]{xy}
\usepackage{graphicx}

\usepackage{graphicx,color,url,hyperref}
\usepackage{epsfig}
\fancyhf{}
\setlength{\parindent}{0pt}
\setlength{\parskip}{5pt plus 1pt}
\setlength{\headheight}{13.6pt}

\newcommand{\NN}{\mathbf N}
\newcommand{\RR}{\mathbf R}
\newcommand{\CC}{\mathbf C}
\newcommand{\ZZ}{\mathbf Z}
\newcommand{\ZZn}[1]{\ZZ/{#1}\ZZ}
\newcommand{\QQ}{\mathbf Q}
\newcommand{\nn}{\mathbb N}
\newcommand{\rr}{\mathbb R}
\newcommand{\cc}{\mathbb C}
\newcommand{\zz}{\mathbb Z}
\newcommand{\zzn}[1]{\zz/{#1}\zz}
\newcommand{\qq}{\mathbb Q}
\newcommand{\calM}{\mathcal M}
\newcommand{\latex}{\LaTeX}
\newcommand{\tex}{\TeX}
\newcommand{\dd}{{\rm d}}
\newcommand{\sm}{\setminus} 

\title{Homework 2 Solutions}
\author{Sunny Lee}
\date{September 10, 2020}

\pagestyle{fancy}
\fancyhf{}
\rhead{9.16.2020}
\lhead{Sunny Lee}
\chead{Math 2550 HW2}
\rfoot{Page \thepage}

\begin{document}

\begin{enumerate}
    \item \emph{Solution.} \\
    Assume every even number greater than 2 can be expresssed 
    as a sum of two primes. Let $p_{a}$ and $p_{b}$ be two arbitrary primes. Then, for
    some natural number k, $p_{a} + p_{b} = 2k$. Let $p_{c}$ be an arbitrary prime number. 
    Then, $p_{c}$ is either 2 or odd. If $p_{c}$ is odd, $p_{c}$ can be written 
    as $2m+1$ for some natural number m. Then, $p_{a} + p_{b} + p_{c} = 2k + 2m + 1$. 
    Since the integers are closed under addition, $k+m$ is an integer. Therefore, 
    $p_{a} + p_{b} + p_{c} = 2(k+m) + 1$ which is an odd number. 

    \item \emph{Solution.}\\
    Suppose there exist integers $n$ and $m$ such that $21n + 6m = 5$. Then, since both 
    $21$ and $6$ are divisible by $3$, by distribution, $21n + 6m = 3(7n + 2m)$. Since 
    the integers are closed under addition and multiplication, $7n + 2m$ is an integer.
    Since $7n + 2m$ is an integer, this number produces a multiple of three. However, 
    5 is not a multiple of 3. Therefore, there do not exist integers $n, m$  such that
    $21n + 6m = 3(7n + 2m) = 5$. 

    \item \emph{Solution.} \\
    Let $c, a \in \zz$, $c > 0$, $a$ is odd, $c$ divides $a$ and $c$ divides $a+2$. Then, 
    $a = 2k + 1$ for some $k \in \zz$. Since $c$ divides $a$ and $a+2$, $cm = a$
    and $cn = a+2$ for some numbers $m, n \in \zz$. 
    
    \item \emph{Solution.}
    Assume that $m$ is not a multiple of 5. Then, $m = 5d + 1$ or $5d + 2$ or $5d + 3$ or $5d + 4$. 
    \begin{enumerate}
        \item Case 1: If $m = 5d + 1$, then $m^2 = (5d + 1)^2 = 5(5d^2+2d)+1$
        \item Case 2: If $m = 5d + 2$, then $m^2 = (5d + 2)^2 = 5(5d^2+4d)+4$
        \item Case 3: If $m = 5d + 3$, then $m^2 = (5d + 3)^2 = 5(5d^2+6d)+9$
        \item Case 4: If $m = 5d + 4$, then $m^2 = (5d + 4)^2 = 5(5d^2+8d)+16$
    \end{enumerate}
    Thus, if $m$ is not a multiple of $5$, then $m^2$ is not a multiple of $5$. Therefore, 
    if $m^2$ is a multiple of five, then $m$ must also be a multiple of five. 

    \item \emph{Solution.} 
    Assume, to the contrary, that $\sqrt{5}$ was a rational number. Then, $\sqrt{5} = \frac{m}{n}$
    for two integers $m, n \in \zz$. Assume $m$ and $n$ are in lowest terms. Then, $5 = \frac{m^2}{n^2}$
    and therefore, $5n^2 = m^2$. Since $m^2$ is a multiple of $5$, by the previously 
    proven theorem, $m$ must also be a multiple of five. Therefore, $m = 5d$ for some 
    $d \in \zz$. Then, since $m = 5d$, $m^2 = 25d^2$. Thus, $5n^2 = 25d^2$. By cancellation, 
    $n^2 = 5d^2$. Thus, $n^2$ is a multiple of five, and by the previous theorem, $n$ must 
    also be a multiple of five. Since n and m are both multiples of five, this contradicts
    the assumption that m and n are in lowest terms. Thus, we reach a contradiction and we 
    conclude that $\sqrt{5}$ cannot be a rational number.
\end{enumerate}

\end{document}