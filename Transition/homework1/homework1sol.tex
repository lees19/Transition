\documentclass{article} 

\usepackage{fancyhdr}
\usepackage[english]{babel}

\usepackage{fullpage}
\usepackage[margin = .75 in]{geometry}
\usepackage[leqno]{amsmath}
\usepackage{amsfonts}
\usepackage{amssymb}
\usepackage{amsthm}
\usepackage{amssymb}
\usepackage[all]{xy}
\usepackage{graphicx}

\usepackage{graphicx,color,url,hyperref}
\usepackage{epsfig}
\fancyhf{}
\setlength{\parindent}{0pt}
\setlength{\parskip}{5pt plus 1pt}
\setlength{\headheight}{13.6pt}

\newcommand{\NN}{\mathbf N}
\newcommand{\RR}{\mathbf R}
\newcommand{\CC}{\mathbf C}
\newcommand{\ZZ}{\mathbf Z}
\newcommand{\ZZn}[1]{\ZZ/{#1}\ZZ}
\newcommand{\QQ}{\mathbf Q}
\newcommand{\nn}{\mathbb N}
\newcommand{\rr}{\mathbb R}
\newcommand{\cc}{\mathbb C}
\newcommand{\zz}{\mathbb Z}
\newcommand{\zzn}[1]{\zz/{#1}\zz}
\newcommand{\qq}{\mathbb Q}
\newcommand{\calM}{\mathcal M}
\newcommand{\latex}{\LaTeX}
\newcommand{\tex}{\TeX}
\newcommand{\dd}{{\rm d}}
\newcommand{\sm}{\setminus} 

\title{Homework 1 Solutions}
\author{Sunny Lee}
\date{September 10, 2020}

\pagestyle{fancy}
\fancyhf{}
\rhead{9.16.2020}
\lhead{Sunny Lee}
\chead{Math 2550 HW1}
\rfoot{Page \thepage}

\begin{document}
    
    \begin{enumerate}
        \item Which of the following are propositions? 
              Give truth values for each proposition.
            \begin{enumerate}
                \item Jack Nicholson was born in Honolulu.
                    \vskip1em
                    \emph{Solution.} This is a proposition. False. 
                    \vskip1em
                \item What time is the train arriving?
                    \vskip1em
                    \emph{Solution.} This is not a proposition. There is no truth value.
                    \vskip1em
                \item 2 + 2 = 4
                    \vskip1em
                    \emph{Solution.} This is a proposition. True.
                    \vskip1em
                \item 2 + 2 = x
                    \vskip1em
                    \emph{Solution.} This is not a proposition. 
                    \vskip1em
            \end{enumerate}
        \item For each pair of statements, determine whether the conjunction P $\wedge$ Q and the disjunction P $\vee$ Q are true
            \begin{enumerate}
                \item $P: \pi < 4, Q: e < 2$
                    \vskip1em
                    \emph{Solution.}
                    \begin{table}[ht]
                        \begin{center}
                            \begin{tabular}{c|c|c|c} 
                            P & Q & P $\wedge$ Q & P $\vee$ Q \\ \hline
                            T & F & F & T\\
                            \end{tabular}
                        \end{center}
                    \end{table}
                    \vskip1em
                \item P: Germany is the capital of Sweden, Q: The prime divsors of 15 are 2 and 5.
                    \vskip1em
                    \emph{Solution.} 
                    \begin{table}[ht]
                        \begin{center}
                            \begin{tabular}{c|c|c|c} 
                            P & Q & P $\wedge$ Q & P $\vee$ Q \\ \hline
                            F & F & F & F\\
                            \end{tabular}
                        \end{center}
                    \end{table}
                    \vskip1em
                \item P: Mexico is south of Alaska, Q: 1738 is a multiple of 11
                    \vskip1em
                    \emph{Solution.}
                    \begin{table}[ht]
                        \begin{center}
                            \begin{tabular}{c|c|c|c} 
                            P & Q & P $\wedge$ Q & P $\vee$ Q \\ \hline
                            T & T & T & T\\
                            \end{tabular}
                        \end{center}
                    \end{table}  
                    \vskip1em
            \end{enumerate}
        \pagebreak
        \item Make a truth table for each of the following propositional forms.
            \begin{enumerate}
                \item $(P \vee Q) \wedge \neg R$
                    \vskip1em
                    \emph{Solution.} 
                    \begin{table}[ht]
                        \begin{center}
                            \begin{tabular}{c|c|c|c|c|c} 
                            $P$ & $Q$ & $R$ & $\neg R$ & $P \vee Q$ & $(P \vee Q) \wedge \neg R$\\ \hline
                            T & T & T & F & T &F\\
                            T & T & F & T & T &T\\
                            T & F & T & F & T &F\\
                            T & F & F & T & T &T\\
                            F & T & T & F & T &F\\
                            F & T & F & T & T &T\\
                            F & F & T & F & F &F\\
                            F & F & F & T & F &F\\
                            \end{tabular}
                        \end{center}
                    \end{table}
                    \vskip1em
                \item $Q \vee (P\vee \neg Q)$
                    \vskip1em
                    \emph{Solution.} 
                    \begin{table}[ht]
                        \begin{center}
                            \begin{tabular}{c|c|c|c|c} 
                            $P$ & $Q$ & $\neg Q$ & $P\vee \neg Q$ & $Q \vee (P\vee \neg Q)$\\ \hline
                            T & T & F & T & T \\
                            T & F & T & T & T \\
                            F & T & F & F & T \\
                            F & F & T & T & T \\
                            \end{tabular}
                        \end{center}
                    \end{table}
                    \vskip1em
                \item $P \Rightarrow (Q \wedge P)$
                    \vskip1em
                    \emph{Solution.} 
                    \begin{table}[ht]
                        \begin{center}
                            \begin{tabular}{c|c|c|c} 
                            $P$ & $Q$ & $Q \wedge P$ & $P \Rightarrow (Q \wedge P)$\\ \hline
                            T & T & F & F \\
                            T & F & T & T \\
                            F & T & F & T \\
                            F & F & T & T \\
                            \end{tabular}
                        \end{center}
                    \end{table}
                    \vskip1em
                \pagebreak
                \item $(P \vee Q) \Rightarrow (P \wedge Q)$
                    \vskip1em
                    \emph{Solution.} 
                    \begin{table}[ht]
                        \begin{center}
                            \begin{tabular}{c|c|c|c|c} 
                            $P$ & $Q$ & $P \vee Q$ & $P \wedge Q$ & $(P \vee Q) \Rightarrow (P \wedge Q)$\\ \hline
                            T & T & T & T &T\\
                            T & F & T & F &F\\
                            F & T & T & F &F\\
                            F & F & F & F &T\\
                            \end{tabular}
                        \end{center}
                    \end{table}
                    \vskip1em
            \end{enumerate}
        \item Suppose that $P$ is equivalent to $Q$ and $R$ is equivalent to $S$. 
              Explain why it must be the case that $P \vee S$
              is equivalent to $Q \vee R$. 
            \vskip1mm
            \emph{Solution.} 
                If $P$ is equivalent to $Q$ and $S$ is equivalent to $R$ then, 
                you can substitute the proposition $P$ with $Q$ and $S$ with $R$ 
                and vice versa. For example with, $P \vee S$, you can substitute
                $P$ with $Q$ and as $P$ is equivalent to $Q$, the proposition
                $Q \vee S$ will still hold the same truth table, therefore equivalent
        \item Identify the antecedent and consequent for each of the following conditional sentences. 
              Assume that a, b, and f represent some fixed sequence, integer,
              or function, respectively
            \begin{enumerate}
                \item If the moon is made of cheese, then 8 is an irrational number.
                \vskip1em
                \emph{Solution.} \\
                    Antecedent: the moon is made of cheese \\
                    Consequent: 8 is an irrational number
                \vskip1em
                \item $b$ divides 25 only if $b$ divides 100.
                \vskip1em
                \emph{Solution.} \\
                    Antecedent: $b$ divides 100 \\
                    Consequent: $b$ divides 25
                \vskip1em
                \item $f$ is differentiable if $f$ is continuous.
                    \vskip1em
                    \emph{Solution.} \\
                        Antecedent: $f$ is continuous\\
                        Consequent: $f$ is differentiable
                    \vskip1em
                \item The fish bite whenever the moon is full.
                    \vskip1em
                    \emph{Solution.} \\
                        Antecedent: The moon is full\\
                        Consequent: The fish will bite
                    \vskip1em
            \end{enumerate}
        \item Which of the following propositions are true?
            \begin{enumerate}
                \item A parallelogram is 5 sided iff 14 is prime.
                    \vskip1em
                    \emph{Solution.} 
                        True
                \item $\forall x, y, z \in \zz$, $xy = xz$ only if $y = z$
                    \vskip1em    
                    \emph{Solution.} %help! ask question about this one
                        True 
                \item $\forall x \in \rr, x^2 \geq 0 $ if $ x > 0.$
                    \vskip1em    
                    \emph{Solution.} 
                        True.
            \end{enumerate}
        \item Give, if possible, an example of a true conditional sentence for which
            \begin{enumerate}
                \item the converse is false.
                    \vskip1em
                    \emph{Solution.} 
                        If the United States is in Europe, then Europe is a continent
                \item the contrapositive is true.
                    \vskip1em
                    \emph{Solution.} 
                        If $1 \in \nn$ then $2 \in \nn$
            \end{enumerate}
        \item Give, if possible, an example of a false conditional sentence for which
            \begin{enumerate}
                \item the converse is false.
                    \vskip1em
                    \emph{Solution.} 
                        Not possible, as the only way a condition is false is if the
                        antecedent is true and the consequent is false. By taking the 
                        converse, that would result in the antecedent being false and
                        the consequent being true which makes the conditional true.
                \item the contrapositive is true.
                    \vskip1em
                    \emph{Solution.}
                        If $1 \in \nn$ then $2.4 \in \nn$ 
            \end{enumerate}
    \end{enumerate}

\end{document}