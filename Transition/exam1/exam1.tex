\documentclass[10pt]{exam}
\firstpageheader{Math 2550 Fall 2020}{Exam 1}{Instructor:  J. Haga}
\usepackage{amsfonts}
\usepackage{amsmath}
\usepackage{amssymb, graphicx, enumerate, mathrsfs}


\begin{document}


\newcommand{\dd}{\textrm{d}}
\newcommand{\NN}{\mathbb N}
\newcommand{\CC}{\mathbb C}
\newcommand{\QQ}{\mathbb Q}
\newcommand{\ZZ}{\mathbb Z}
\newcommand{\RR}{\mathbb R}
\newcommand{\proofdone}{\hfill $\square$}

\vskip 0.5cm
\begin{enumerate}
    \item 
    \begin{enumerate}
        \item 
        The statement when $n = 0$ is true since: 
        \begin{gather}
            P(0) = c_0x^0 = c_0
        \end{gather}
        Since $P(0) = c_0$ and $c_0$ is a constant, $P(0)$ is a continuous function.
        \item 
        Assume $P(N)$. Now consider $P(N+1)$. Then, 
        \begin{gather*}
            P(N+1) = c_0 + c_1x + c_2x^2 + \dots + c_Nx^N + c_{N+1}x^{N+1}\\
            P(N+1) = c_0 + x(c_1 + c_2x + \dots + c_Nx^{N-1} + c_{N+1}x^{N})
        \end{gather*}
        Since $c_1 + c_2x + \dots + c_Nx^{N-1} + c_{N+1}x^{N}$ is a $N^{th}$ degree
        polynomial with real coefficients, $x(c_1 + c_2x + \dots + c_Nx^{N-1} + c_{N+1}x^{N})$
        is a continuous function. Since $c_0$ is a constant function, $c_0 + x(c_1 + c_2x + \dots + c_Nx^{N-1} + c_{N+1}x^{N})$ 
        is a continuous function. 
        \item 
        Thus, by the Principle of Mathematical Induction, all polynomial functions with real 
        coefficients are continuous functions. 
    \end{enumerate}
    
    \pagebreak
    \item 
    Assume the Intermediate Value Theorem, and assume $f(x) = x^2 - 2$ on the interval 
    $[a, b]$. By the theorem proved by question 1, $f(x)$ is continuous, since $x^2$ and 
    $-2$ are both continuous functions, and the sum of two continuous functions is continuous.
    Then, since $f(1) = -1 < 0$ and $f(2) = 2 > 0$, by the Intermediate Value Theorem, 
    we know $\exists c \in (a, b) \text{ such that} f(c) = 0$. Therefore, at some point $c$, 
    $f(c) = c^2 - 2 = 0$. Solving for $c$, we get $c^2 = 2$ and thus, $c = \sqrt{2}$. 
    To prove $\sqrt{2}$ is irrational, assume to the contrary, $\sqrt{2}$ is rational. 
    Then, $\sqrt{2}$ can be written by two integers $p$ and $q$ such that 
    $\sqrt{2} = \frac{p}{q}$ and $\frac{p}{q}$ are in lowest terms. Then, $2 = \frac{p^2}{q^2}$
    Thus, $2q^2 = p^2$ and from a previously proven theorem, if $p^2$ is even, $p$ is also even since 
    if $p$ was odd, $p^2$ would also be odd. Since $p$ is even, $p = 2k$ for some integer k. 
    Since $p = 2k$, $p^2 = 4k^2$. Thus, $2q^2 = 4p^2$. By cancellation, $q^2 = 2p^2$. Since 
    $q^2$ is even, q must also be even. This is a contradiction, since we assumed $p$ and $q$ 
    were in lowest terms. Therefore, $\sqrt{2}$ must be an irrational number. 

    \pagebreak
    %well, the gcd(43, 18) is exactly one. Since gcd(43, 18) = 1 is the smallest 
    %positive linear combination of 43x + 18y, that's gonna lead us to Z
    %if the smallest linear combination is 1, then we can take any multiple of 1
    %aka all of Z, so this is clearly true. 
    \item 
    Case $S \subseteq \ZZ$\\
    Let $x$ be an arbitrary element of $S$. Since $x\in S$, $x$ is some number 
    of the form $43j + 18k$ for some integers $j, k \in \ZZ$. Since $j$ is an integer, 
    $43j$ is an integer and since $k$ is an integer, $18k$ is an integer by closure. 
    Since $43j$ and $18k$ are integers, their sum, $43j + 18k$ is an integer. Since
    $x$ was arbitrary, $S \subseteq \ZZ$. 
    \vskip1em
    %pick arbitrary, 
    Case $\ZZ \subseteq S$\\
    Let $y$ be an arbitrary element in $\ZZ$. Since $43$ and $18$ are relatively 
    prime, there exist two integers $n$ and $m$, such that $1 = 43n + 18m$. 
    Since $1 = 43n + 18m$, $y = 43ny + 18my$. Since $y$ was arbitrary, it follows
    that $\ZZ \subseteq S$. 
    %Suppose x is an arbitrary element in Z. Then, x = 1*x. Since 1 is the smallest 
    %linear combination of 43 and 18, 1 = 43j + 18k. Since 1 = 43j + 18k, 
    %(43j + 18k) * x = 43xj + 18xk = x. Since x is an element of Z, 
    %xj and xk are integers. Since 43 and xj are integers, 43xj is an integer. 
    %Since 18 and xk are integers, 18xk is an integer. Since 43xj and 18xk are 
    %integers, 43xj + 18xk is an integer. Since 43xj + 18xk = x and x was arbitrary, 
   % Z is a subset of S. 

    Thus, since S is a subset of Z and Z is a subset of S, S = Z
    \pagebreak
    %so, because these are sets, to prove an equivalence, we find the interval 
    %that we think is correct, then prove that they're both subsets of each other
    \begin{enumerate}
        \item 
        Case $\bigcap_{r\in \RR^+}A_r \subseteq \varnothing$\\
        Assume for every real number, the set $A_r$ is the interval $[\frac{r+1}{r}, \frac{5r + 3}{r})$.
        If $r = 2$ then, $A_2 = [1.5, 6.5)$. If $r = \frac{1}{6}$ then $A_{\frac{1}{6}} = [7, 23)$. 
        Thus we have found two $A_r$ which are disjoint. Thus since the intersection 
        of these two sets is the null set, the intersection of every $A_r$ must be the 
        null set. Thus the intersection of all of the $A_r$ is a subset of the null set.
        \vskip1em

        Case $\varnothing \subseteq \bigcap_{r\in \RR^+}A_r$\\
        Since the null set is a subset of every set, we find that the null set is must be 
        a subset of the infinite intersection of the $A_r$. Thus, the null set is a subset
        of the infinite intersection of $A_r$. 
        \vskip1em
        
        Thus, since the infinite intersection is a subset of the null set and the null set
        is a subset of the infinite intersection, $\bigcap_{r\in \RR^+}A_r = \varnothing$
        \pagebreak
        \item 
        Case $\bigcup_{r\in \RR^+}A_r \subseteq (1, \infty)$\\
        Let $x$ be an arbitrary element of $\bigcup_{r\in \RR^+}$. 
        Since $x\in \bigcup_{r\in \RR^+}A_r$, $x$ must be in at least one of the $A_r$. 
        Since $x\in A_r$ for some positive real number $r$, 
        $x\in [\frac{r+1}{r}, \frac{5r+3}{r}]$, thus $\frac{r+1}{r} \leq x$.
        Since $r$ is a positive real number, $r < r+1$. Since $r<r+1$, $1 < \frac{r+1}{r}$. 
        Since $1 < \frac{r+1}{r} \leq x$, $1 < x$. Since $x$ was arbitrary,
        $\bigcup_{r\in \RR^+}A_r \subseteq (1, \infty)$


        %Let x be an arbitrary element of the infinite union of the $A_r$. Then, 
        %since x is in the union of $A_r$, x is in at least one of the $A_r$. Since 
        %x is in at least one of the $A_r$, x is in at least one of the intervals 
        %$[\frac{r+1}{r}, \frac{5r+3}{r}]$. Since x is in at least one of these intervals,
        %x is greater than or equal to $\frac{r+1}{r}$ and less than or equal to 
        %$\frac{5r+3}{r}$. Since r is any real number, if we let r approach infinity, 
        %$\frac{r+1}{r}$ approaches 1. If we let r approach 0, $\frac{5r+3}{r}$ approaches
        %infinity. Thus, x is greater than 1 and less than infinity. Thus, x must be in 
        %the interval (1, inf). Since x was arbitrary, the infinite union of the $A_r$ 
        %is a subset of (1, inf). 

        \vskip1em
        Case $(1, \infty) \subseteq \bigcup_{r\in \RR^+ A_r}$\\
        Let $y$ be an arbitrary element of the interval $(1, \infty)$. We want to show
        $y\in [\frac{r+1}{r}, \frac{5r+3}{r}]$. Let $y = \frac{r+1}{r}$. Then, by algebra, 
        $r = \frac{1}{(y-1)}$. By closure, $r$ is a real number. 
        Since $y > 1$, $y-1 > 0$. Thus, $\frac{1}{y-1}$ is positive. Thus, we have found a 
        positive real number $r = \frac{1}{(y-1)}$ for which $y\in A_r$. Since $y$ was arbitrary, 
        $(1, \infty) \subseteq \bigcup_{r\in \RR^+ A_r}$. 
        
        %Let x be an arbitrary element of the interval (1, inf). Since x is in (1, inf), 
        %x is greater than 1 and less than inf. Since $\frac{r+1}{r}$ approaches 1 as 
        %r approaches infinity, 1 < $\lim_{r\rightarrow \infty}\frac{r+1}{r} \leq$ x. 
        %Since as r approaches zero, $\frac{5r+3}{r}$ approaches infinity, 
        %x < $\lim_{r\rightarrow 0} \frac{5r+3}{r}$ < $\infty$. Since 
        %$\lim_{r\rightarrow \infty}\frac{r+1}{r} \leq x \leq \lim_{r\rightarrow 0} \frac{5r+3}{r}$, 
        %x must be in at least one of the $A_r$. 

        Thus, since each subset is a subset of the other, $\bigcup_{r\in \RR^+} = (1, \infty)$
    \end{enumerate}




    
\end{enumerate}

    
\end{document}