\documentclass{article} 

\usepackage{fancyhdr}
\usepackage[english]{babel}

\usepackage{fullpage}
\usepackage[margin = .75 in]{geometry}
\usepackage[leqno]{amsmath}
\usepackage{amsfonts}
\usepackage{amssymb}
\usepackage{amsthm}
\usepackage{amssymb}
\usepackage[all]{xy}
\usepackage{graphicx}

\usepackage{graphicx,color,url,hyperref}
\usepackage{epsfig}
\fancyhf{}
\setlength{\parindent}{0pt}
\setlength{\parskip}{5pt plus 1pt}
\setlength{\headheight}{13.6pt}

\newcommand{\NN}{\mathbf N}
\newcommand{\RR}{\mathbf R}
\newcommand{\CC}{\mathbf C}
\newcommand{\ZZ}{\mathbf Z}
\newcommand{\ZZn}[1]{\ZZ/{#1}\ZZ}
\newcommand{\QQ}{\mathbf Q}
\newcommand{\nn}{\mathbb N}
\newcommand{\rr}{\mathbb R}
\newcommand{\cc}{\mathbb C}
\newcommand{\zz}{\mathbb Z}
\newcommand{\zzn}[1]{\zz/{#1}\zz}
\newcommand{\qq}{\mathbb Q}
\newcommand{\calM}{\mathcal M}
\newcommand{\latex}{\LaTeX}
\newcommand{\tex}{\TeX}
\newcommand{\dd}{{\rm d}}
\newcommand{\sm}{\setminus} 

\title{Homework 2 Solutions}
\author{Sunny Lee}
\date{September 10, 2020}

\pagestyle{fancy}
\fancyhf{}
\rhead{9.16.2020}
\lhead{Sunny Lee}
\chead{Math 2550 HW2}
\rfoot{Page \thepage}

\begin{document}
    \begin{enumerate}
        \item \emph{Solution.}
            \begin{enumerate}
                \item Everyone has sent an email to someone
                \item Everyone has sent an email to everybody
                %there exists a p st if every q has sent an email to p, 
                %and if every t has sent an email to r, then r is p 
                \item Every q has sent an email to some p and for every r, if every t has sent an email to r, then r is p
            \end{enumerate}
        \item \emph{Solution.} 
            $\exists!p$ $(\forall q S(q, p))$
        \item \emph{Solution.}
            Suppose $x$ and $y$ are even. Then, from the definition of even, we can 
            rewrite $x$ and $y$ in the form $x = 2n$ and $y = 2m$ for some integers
            $n$ and $m$. Then, plugging in $2n$ and $2m$ we are left with the new 
            equation $5(2n) - 3(2m)^2 + 3$. Since $3 = 2 + 1$, we can rewrite the 
            equation as $10n - 12m^2 + 2 + 1$. From here, we can factor out a 2 
            from the first three terms and we are left with an equation of the form
            $2(5n - 6m^2 + 1) + 1$. Let $s = (5n - 6m^2 + 1)$ Since $n$ and $m$ are integers, and since $\zz$ is 
            closed under addition and multiplication, $s = (5n - 6m^2 + 1)$ is also an integer.
            Therefore, since $5x - 3y^2 + 3 = 2(5n - 6m^2 + 1) + 1 = 2s + 1$, 
            we conclude $5x - 3y^2 + 3$ must be an odd integer if $x$ and $y$ are even. 
        \item \emph{Solution.} 
            Suppose $a, b$ and $c$ were integers. Assume $a$ divides $b$ and $c$ divides
            $d$. In that case, if we factor out $b$ from our number $2bd + b^2d^2$
            we are left with $b(2d + bd^2)$. Since $a$ divides $b$, we know this whole 
            number is divisible by $a$. We can also factor out a $d$ from our number: 
            $d(2b + b^2d)$. Since $c$ divides $d$, we also know our number can be divided by 
            $c$. Therefore, if we factor out both $b$ and $d$ from our number, we can see
            that the number must be divisible by both $a$ and $c$. Thus, the number $2bd + b^2d^2$ 
            is divisible by $ac$
        \item \emph{Solution.} 
            Suppose $x \in \rr$ and $x$ is arbitrary. Assume $3-x \leq$ 0.
            Since $3-x \leq 0$ we can add x from both sides to get $3 \leq x$. Squaring both 
            sides of the inequality, $9 \leq x^2$ and cubing both sides of the assumed inequality, 
            $27 \leq x^3$. Adding the two inequalities, $27 + 9 = 36 \leq x^3 + x^2$. 
            Therefore, $1 < 36 \leq x^3 + x^2$. Thus, if $8-x^3-x^2 \geq 7$, then $3-x > 0$. 
        \item \emph{Solution.} 
            Assume $x \in \zz$ and is an arbitrary integer. Assume $x$ is odd. Then, by definition, 
            $x$ can be written in the form $x = 2(2k) + 1 = 4k + 1$ for some integer $k$. Then, by algebra, 
            $x^2 - 1 = (4k+1)^2 - 1 = (16k^2+8k+1)-1 = 8(2k^2+k)+1-1 = 8(2k^2+k)$. Since 
            $x^2-1 = 8(2k^2+k)$ for some arbitrary integer $k$, and because of the closure properties of $\zz$,
            $x^2-1$ is divisible by 8. Thus, we conclude if 8 does not divide $x^2-1$ then $x$ is even. 
        \item \emph{Solution.}
            \begin{enumerate}
                \item Assume $(x, y) \in \rr^2$ and $(x-1)^2 + (y-4)^2 \leq 9$. Then, this equation can be rewritten as 
                      the distance between the points $(1, 4)$ and $(x, y)$ less than or equal to 3: $d((1, 4), (x, y)) \leq 3$.
                      We also know the distance between the two centers of the circles is: $d((1, 4), (2, 3)) = \sqrt{2}$
                      Therefore, the distance between the two centers and the distance from $(1, 4)$ to any point $(x, y)$
                      satisfying $d((x, y), (1, 4)) \leq 3$ is at most $\sqrt{2} + 3$, which is less than or equal to $5$: $d((2, 3), (1, 4)) + d((1, 4), (x, y)) \leq \sqrt{2} + 3 \leq 5$.
                      Therefore, if given any $(x, y) \in \rr$ which satisfies $(x-1)^2 + (y-4)^2 \leq 9$, must also satisfy $(x-2)^2 + (y-3)^2 \leq 25$
                \item With two circles of radius 3 and 5, we know that if we place the centers of the two circles at certain points, 
                    we find that the circle with a larger radius can completely cover the one with the samller radius. Therefore, if 
                    the two centers of the two circles are within the difference between the two radii, we find that every point within
                    the circle of smaller radius must be contained within the circle of the larger radius. 
            \end{enumerate}
            
        \item \emph{Solution.} 
            Assume $|m| > |n|$ then, there are four cases: either both are positive, both are 
            negative, $m$ is negative and $n$ is positive and $m$ is positive and $n$ is negative. 
            \begin{enumerate}
                \item Case 1: $m$ and $n$ are both positive\\
                    If $m$ and $n$ are both positive, then $|m| = m$ and $|n| = n$. Therefore, $m > n$. 
                    Squaring both sides: $m^2 > n^2$ and subtracting $n^2$ from both sides: $m^2-n^2>0$
                \item Case 2: m and n are both negative\\
                    If $m$ and $n$ are both negative, then $|-m| = m$ and $|-n| = n$. Therefore, $m > n$. 
                    Squaring both sides: $m^2 > n^2$ and subtracting $n^2$ from both sides: $m^2-n^2>0$
                \item Case 3: m is negative, n is positive\\
                    If $m$ is negative and $n$ is positive, then $|-m| = m$ and $|n| = n$. Therefore, $m > n$. 
                    Squaring both sides: $m^2 > n^2$ and subtracting $n^2$ from both sides: $m^2-n^2>0$
                \item Case 4: m is positive, n is negative\\
                    If $m$ is positive and $n$ is negative, then $|m| = m$ and $|-n| = n$. Therefore, $m > n$. 
                    Squaring both sides: $m^2 > n^2$ and subtracting $n^2$ from both sides: $m^2-n^2>0$
            \end{enumerate}     
            Since every case prodcues the same result, namely $m^2 - n^2 > 0$, we conclude if $|m| > |n|$, then 
            $m^2 - n^2 > 0$
            

    \end{enumerate}
\end{document}